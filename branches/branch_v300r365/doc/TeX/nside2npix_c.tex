

\sloppy


\title{\healpix C Subroutines Overview}
\docid{nside2npix} \section[nside2npix]{ }
\label{csub:nside2npix}
\docrv{Version 1.0}
\author{E. Hivon}
\abstract{This document describes the \healpix C subroutine NSIDE2NPIX.}

\begin{facility}
{Function to provide the number of pixels $\npix$ over the full sky corresponding
to resolution parameter $\nside$. 
}
{src/C/subs/chealpix.c}
\end{facility}

\begin{Cfunction}
{long nside2npix(const long nside)}
\end{Cfunction}

\begin{arguments}
{
\begin{tabular}{p{0.3\hsize} p{0.05\hsize} p{0.1\hsize} p{0.45\hsize}} \hline  
\textbf{name\&dimensionality} & \textbf{kind} & \textbf{in/out} & \textbf{description} \\ \hline
                   &   &   &                           \\ %%% for presentation
nside & long & IN & the $\nside$ parameter of the map. \\
\thedocid  & long & OUT & returns the number of pixels $\npix$ of the map $\npix=12\nside^2$.\\
\end{tabular}
}
\end{arguments}

\begin{example}
{
npix= nside2npix(256);  \\
}
{
Returns the number of \healpix pixels (786432) for the resolution
parameter 256.
}
\end{example}
% \begin{related}
%   \begin{sulist}{} %%%% NOTE the ``extra'' brace here %%%%
%   \item[\htmlref{npix2nside}{sub:npix2nside}] returns resolution parameter corresponding to the number of pixels.
% %   \item[pix2xxx] conversion between pixel index and position on the sky.
%   \end{sulist}
% \end{related}

\begin{related}
  \begin{sulist}{} %%%% NOTE the ``extra'' brace here %%%%
%%   \item[\htmlref{neighbours\_nest}{sub:neighbours_nest}] find neighbouring pixels.
  \item[\htmlref{ang2vec}{csub:ang2vec}] converts $(\theta,\phi)$ spherical coordinates into $(x,y,z)$ cartesian coordinates.
  \item[\htmlref{vec2ang}{csub:vec2ang}] converts $(x,y,z)$ cartesian coordinates into $(\theta,\phi)$ spherical coordinates.
%   \item[\htmlref{nside2npix}{csub:nside2npix}] converts number of full sky
% pixels $\npix$ into resolution parameter $\nside$
  \item[\htmlref{npix2nside}{csub:npix2nside}] converts $\nside$ into number of
full sky pixels $\npix$.
  \end{sulist}
\end{related}

\rule{\hsize}{2mm}

